%Chiamata classi e impostazioni fuori documento. 

\documentclass{beamer}
\usepackage[italian]{babel} 
\usepackage[latin1]{inputenc} 
\usepackage[T1]{fontenc}

\title{Analisi della struttura sanitaria della provincia di Ascoli Piceno} 
\author{Enrico Ferretti \\ 
	Tommaso Cicco\\
	Francesco Rombaldoni\\
	} 
\institute{Universit� degli Studi di Urbino "Carlo Bo"} 
\logo{\includegraphics[width=15mm]{Uni}}


\usetheme{AnnArbor} 
%\useoutertheme[right]{sidebar} 
\setbeamercovered{dynamic}

%inizio socumento
\begin{document}


%Prima slide (Copertina)
	\begin{frame} 
		\maketitle 
	\end{frame}

%seconda slide
	\begin{frame} 
		\frametitle{Indice} 
		\tableofcontents 
	\end{frame}

%Primo frame complesso
	\section{Presentazione} 
	\begin{frame} 
		\frametitle{Obbiettivo:} 
	%	\framesubtitle{Sottotitolo} 
	L'obbiettivo dell'analisi � di determinare se la rete delle strutture della provincia di Ascoli Piceno che erogano servizi d'assistenza psichiatrica corrisponde a una delle seguenti strutture ed il cambiamento negli anni 2015, 2017, 2019:  
	
	\bigskip
	
	\begin{enumerate}
		\item Organizzazione diffusa.
		\item Organizzazione centralizzata.
		\item Organizzazione Integrata
	\end{enumerate}

\end{frame}




\end{document}




